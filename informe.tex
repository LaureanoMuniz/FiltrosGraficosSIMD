% ******************************************************** %
%              TEMPLATE DE INFORME ORGA2 v0.1              %
% ******************************************************** %
% ******************************************************** %
%                                                          %
% ALGUNOS PAQUETES REQUERIDOS (EN UBUNTU):                 %
% ========================================
%                                                          %
% texlive-latex-base                                       %
% texlive-latex-recommended                                %
% texlive-fonts-recommended                                %
% texlive-latex-extra?                                     %
% texlive-lang-spanish (en ubuntu 13.10)                   %
% ******************************************************** %


\documentclass[a4paper]{article}
\usepackage[spanish]{babel}
\usepackage[utf8]{inputenc}
\usepackage{charter}   % tipografia
\usepackage{graphicx}
%\usepackage{makeidx}
\usepackage{paralist} %itemize inline

%\usepackage{float}
%\usepackage{amsmath, amsthm, amssymb}
%\usepackage{amsfonts}
%\usepackage{sectsty}
%\usepackage{charter}
%\usepackage{wrapfig}
%\usepackage{listings}
%\lstset{language=C}

% \setcounter{secnumdepth}{2}
\usepackage{underscore}
\usepackage{caratula}
\usepackage{url}
\usepackage{ragged2e}


% ********************************************************* %
% ~~~~~~~~              Code snippets             ~~~~~~~~~ %
% ********************************************************* %

\usepackage{color} % para snipets de codigo coloreados
\usepackage{fancybox}  % para el sbox de los snipets de codigo

\definecolor{litegrey}{gray}{0.94}

\newenvironment{codesnippet}{%
	\begin{Sbox}\begin{minipage}{\textwidth}\sffamily\small}%
	{\end{minipage}\end{Sbox}%
		\begin{center}%
		\vspace{-0.4cm}\colorbox{litegrey}{\TheSbox}\end{center}\vspace{0.3cm}}



% ********************************************************* %
% ~~~~~~~~         Formato de las páginas         ~~~~~~~~~ %
% ********************************************************* %

\usepackage{fancyhdr}
\pagestyle{fancy}

%\renewcommand{\chaptermark}[1]{\markboth{#1}{}}
\renewcommand{\sectionmark}[1]{\markright{\thesection\ - #1}}

\fancyhf{}

\fancyhead[LO]{Sección \rightmark} % \thesection\ 
\fancyfoot[LO]{\small{Nombre Apellido, Nombre Apellido, Nombre Apellido}}
\fancyfoot[RO]{\thepage}
\renewcommand{\headrulewidth}{0.5pt}
\renewcommand{\footrulewidth}{0.5pt}
\setlength{\hoffset}{-0.8in}
\setlength{\textwidth}{16cm}
%\setlength{\hoffset}{-1.1cm}
%\setlength{\textwidth}{16cm}
\setlength{\headsep}{0.5cm}
\setlength{\textheight}{25cm}
\setlength{\voffset}{-0.7in}
\setlength{\headwidth}{\textwidth}
\setlength{\headheight}{13.1pt}

\renewcommand{\baselinestretch}{1.1}  % line spacing

% ******************************************************** %


\begin{document}


\thispagestyle{empty}
\materia{Organización del Computador II}
\submateria{Primer Cuatrimestre de 2019}
\titulo{Trabajo Práctico II}
\subtitulo{Procesamiento de imágenes (SIMD)}
\integrante{Ivo Pajor}{460/19}{ivo_pajor@hotmail.com}
\integrante{Luciana Gorosito}{577/18}{lugorosito0@gmail.com}
\integrante{Laureano Muñiz}{498/19}{lau2000m@hotmail.com}

\maketitle
\newpage

\thispagestyle{empty}
\vfill
\begin{abstract}
En el presente trabajo se describe la problemática de ...
\end{abstract}

\thispagestyle{empty}
\vspace{3cm}
\tableofcontents
\newpage


%\normalsize
\newpage

\section{Introducción}

\justify
El objetivo de este Trabajo Práctico es analizar y comprender el modelo de procesamiento SIMD \textit{(Single Instruction, Multiple Data)} y su relación con la microarquitectura del procesador, mediante la implementación en lenguaje ensamblador de cuatro filtros gráficos: <<\textit{Imagen Fantasma}>>, <<\textit{Color Bordes}>>, <<\textit{Reforzar Brillo}>> y <<\textit{Pixelado Diferencial}>>.
\justify
Más detalladamente, el filtro <<\textit{Imagen Fantasma}>> combina una imagen original con su versión en escala de grises y del doble de tamaño, generando así un  efecto de imagen fantasma sobre la imagen destino. El filtro <<\textit{Color Bordes}>> detecta los bordes de una  imagen y el filtro <<\textit{Reforzar Brillo}>> modifica el brillo de una imagen, aumentándolo en el caso de que  supere al valor del parámetro \textit{umbralSup} y disminuyéndolo en el caso de que esté por debajo del valor  del parámetro \textit{umbralInf}. Por último, el filtro <<\textit{Pixeleado diferencial}>> convierte 4x4 píxeles de una imagen en un solo pixel, generando un efecto pixeleado algunas zonas de la imagen destino. 
\justify
\indent Las implementaciones de estos filtros se realizaron utilizando el set de instrucciones \textbf{SSE} y técnicas de programación vectorial, que permitieron procesar en paralelo de 2 a 4 píxeles, dependiendo del filtro. Posterior a la implementación se realizó un análisis de rendimiento en comparación a las implementaciones en lenguaje C, provistas por la cátedra. Además, se diseñaron dos experimentos motivados en entender las posibles causas de la variación del rendimiento y la limitación de la performance de los algoritmos en el procesamiento de imágenes, que serán detallados en las secciones siguientes.  


\section{Desarrrollo}

\subsection{Implementaciones en ASM}

\subsubsection{Imagen Fantasma}

\subsubsection{Color Bordes}

\subsubsection{Reforzar Brillo}

\subsubsection{Pixeleado Diferencial}




\subsection{Comparación entre implementaciones en ASM y C}

\subsection{Diseño experimental}

\begin{codesnippet}
\begin{verbatim}

struct Pepe {

    ...

};

\end{verbatim}
\end{codesnippet}


\section{Resultados}
%\input{enunciado}

\section{Conclusión}


\end{document}

